\chapter{硬件设计}

在这章中,我们将会讨论基于屏上摄像头的手型识别与交互技术的硬件设计,并说明在我们的硬件方案下,摄像头能有效地捕捉用户正常操作手机时的手部图像,为后续的算法与交互应用设计提供硬件上的支持。

\section{三棱反射镜成像的视角与光路设计}

\begin{figure}
\centering
\includegraphics[width=5in]{figures/FoV.png}
\caption{摄像头视角范围示意图}
\label{fig:FoV}
\end{figure}

我们借鉴了HandSee\cite{Yu:2019:HEF:3290605.3300935}中借助三棱镜的轻量化硬件设计的想法,进行了硬件设计。我们将斜面镀膜的直角三棱镜放置于前置摄像头上方,它将触摸屏上方空间的图像反射至前置摄像头中。为了确保这样的成像方案能够有效地捕捉用户正常操作时的手部行为,该成像方案必须具有较大的视角范围(FoV)。经过计算,在6英寸的智能手机屏幕下,成像的视角范围需达到$60^{\circ}$(水平) $\times$ $35^{\circ}$(垂直),如\ref{fig:FoV}所示,这基于用户正常持握时手部均位于手机手机底部8cm范围内的假设。因此,在$60^{\circ}$ $\times$ $35^{\circ}$的视角范围内,当用户正常持握或正常操作手机(操作手高度小于10cm)时,用户的持握手与操作手均能被较完整地捕获。

\begin{figure}
\centering
\includegraphics[width=5in]{figures/method/guanglu.png}
\caption{前置摄像头成像光路图}
\label{fig:guanglu}
\end{figure}


我们的硬件系统成像的光路图如图\ref{fig:guanglu}所示。由于反射次数与反射面介质不同,区域1、2、3中的成像性质也会有不同(如图\ref{fig:capture}所示),其中区域1为通过一次镜面反射直接进入摄像头的光线,成像质量最为纯净可靠;区域2成像较暗且会混入屏幕光,质量最差;区域3通过三棱镜侧面镜面与三棱镜底面两次反射,成像会有一定的系统色彩偏差。因此,我们主动地抛弃了2、3区域,仅使用区域1的图像作为后续算法的输入。



\section{硬件原型}

\begin{figure}
\centering
\includegraphics[width=5in]{figures/method/capture.png}
\caption{RGB/模拟IR环境下摄像头采集的图像}
\label{fig:capture}
\end{figure}

我们使用了Samsung Galaxy S9+作为实验机型,我们通过3D建模与3D打印技术打印手机外壳用以固定三棱反射镜,在三棱反射镜左侧放置一枚LED灯,并为该LED灯设计了独立的电池盒与开关(如\ref{fig:hardware_design}所示,实物图如\ref{fig:hardware_real})。我们的硬件设计的一个直接的优势是它可以模拟RGB摄像头与IR摄像头(红外摄像头)两种工作模式:在自然光环境下,关闭LED灯,摄像头捕捉自然光环境下的RGB图像;在黑暗环境下打开LED灯,若周围无其他光源,则该LED灯为唯一的主动光源,此时捕获的图像等价于IR摄像头通过窄带滤光片捕获到的由单一红外光源形成的红外图像(如\ref{fig:capture}右图所示)。对于后者,由于此时仅有唯一的主动光源,摄像头捕获到的图像具有背景纯净、质量较高的特点,易于后续的处理,具有更大的潜力。

经过测量,该实验系统的视野范围(FoV)为$56^{\circ}$ $\times$ $33^{\circ}$,水平视野范围可覆盖至手机左右侧边中点之间构成的张角的区域,垂直视野能覆盖正常操作的手部范围,因此能有效地捕捉正常操作时的手部图像。


\begin{figure}
\begin{minipage}[t]{0.5\linewidth}
\centering
\includegraphics[height=1.8in]{figures/method/shell.png}
\caption{硬件设计图}
\label{fig:hardware_design}
\end{minipage}%
\begin{minipage}[t]{0.5\linewidth}
\centering
\includegraphics[height=1.8in]{figures/method/shiwutu.png}
\caption{实物图}
\label{fig:hardware_real}
\end{minipage}
\end{figure}