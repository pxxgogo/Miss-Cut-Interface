误触分类:

1、点击非舒适区域所造成的误触(边缘等):有点击过程,多个触点
2、无点击过程的误触(放口袋里、抓着手机):无点击过程
3、其他信号点造成的干扰(下雨、手是湿的):有点击过程,多个触点

指尖下降检测(形式化)

设$S_k = \{p_{k,1}, p_{k,2}, ... , p_{k,m_k}\}$为第k个视频帧所对应的指尖候选点集,则我们需要检测的是视频流所对应的候选点集序列$S_0, S_1, ... , S_k$中是否存在满足下降运动模式$P$的点的序列。我们首先来定义下降运动模式$P$,我们称点序列$s_p=\{p_{a_i,b_i}|i=1,2,...,m; a_1 < a_2 < ... < a_m\}$满足运动模式$P$当且仅当$$\forall i \in \{1,2,...,m-1\}, 0 < y_{p_{i+1}} - y_{p_i} < \epsilon_1, |x_{p_{i+1}} - x_{p_i}| < \epsilon_2$$

其中$y_{p_{i+1}} - y_{p_i} > 0$保证了下降趋势,$y_{p_{i+1}} - y_{p_i} < \epsilon_1$ 和 $|x_{p_{i+1}} - x_{p_i}| < \epsilon_2$保证了候选点的一致性。

对于相邻的k帧的指尖候选点集$S_k = \{p_{k,1}, p_{k,2}, ... , p_{k,m_k}\}$,设其满足下降运动模式$P$的最长子序列$s_p$长度为$l={max(L(s_p)) \over k} $,则定义置信系数$\lambda={l \over k}={max(L(s_p)) \over k}$。当$\lambda > \lambda_0$,可认为在该帧区间内发生指尖下降行为。

进一步地,可以使用动态规划的策略对问题进行求解:

\begin{algorithm}
\caption{使用动态规划求解最长下降子序列长度}
\begin{algorithmic}
\Require 相邻的k帧的指尖候选点集$S_k = \{p_{k,1}, p_{k,2}, ... , p_{k,m_k}\}$
\Ensure 最长子序列长度

\end{algorithmic}
\end{algorithm}

由于视频帧有一定概率出现模糊,导致指尖候选集的结果错误或不完整,因此需要一定的容错机制,这在算法中通过不强制限定帧间连续来实现。


dynamic menu layout, reinforcement learning

related paper: 29, 249, 


误触识别、被动交互

误触识别:

误触模型:
    (1) 手的模型,见CHI p31和 [1] [2] [3],手的几何建模、舒适区域等,舒适区域外-发生误触概率高
    (2) 指尖下落模型。
任务定义:
    输入:用户触发触摸操作时,handsee camera捕获的视频帧(和相邻的帧)及触摸点信息(坐标、面积)
    输出:对于每个触摸点,给出accept/reject的分类

使用场景:
    边缘误触、湿手操作、口袋误触、躺着

实验设计:
    拇指操作区域建模:测量志愿者的手型(手指长度),让志愿者以不同的方式单手持握(最舒适、完全抓握(四指露出)、
    手机放置在四指上(四指不露出)、握底部),用拇指分别以最舒适、最大限度两种方式画弧,进行记录(包括记录弧线轨迹、摄像头视频、所有触摸点信息)
    点击实验(极限点击与湿手点击):极限点击/湿手点击时,有x%概率产生多余触点(或错误),在这些情况下,误触算法能有效召回(纠正)y%的样本
    使用误触检测算法后用户体验的survey

被动交互:

使用场景:
    靠近唤醒、hand-around不熄屏、
todo:
    hand-on/hand-off/hand-around classifiler



unintentional input, limited reachability
    

[1] Investigating how the hand interacts with different mobile phones
[2] Understanding Users’ Touch Behavior on Large Mobile Touch-Screens and Assisted Targeting by Tilting Gesture
[3] Modeling the Functional Area of the Thumb on Mobile Touchscreen Surfaces

技术:可以focus在提高指尖检测的准确率与鲁棒性上,SSD + MeanShift(?)

喻老师修改意见:
标题
结构不好,算法评测与交互评测的实验分开,分别放在每部分后面
算法评测要写明评测指标
硬件设计另开一章,放在算法前

中文论文摘要后面要写贡献

“流程”这个词不好

实验条件可以写得更具体