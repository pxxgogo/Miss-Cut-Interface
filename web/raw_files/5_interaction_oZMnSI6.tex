\chapter{基于智能手机手部全感知的被动交互技术}

\section{误触检测}
误触是人们使用智能手机时频繁出现的问题,受限与屏幕大小、屏幕传感原理等因素,误触在人们使用智能手机时频繁发生,极大地影响了智能手机的使用效率与使用体验。

我们将常见的误触场景总结为以下四类:
\begin{itemize}
    \item 用户有意识的手部点击行为引起的非目标位置的电容信号造成的误触,例如曲面屏幕上边缘误触、非舒适点击引起的误触等;
    \item 用户无意识的手部非点击行为引起的电容信号造成的误触,例如用户使用手机时(例如看视频)的无意识触碰;
    \item 其他导电介质触碰产生的电容信号造成的误触,例如下雨或者湿手使用手机时屏幕上的水对输入信号的干扰、放在口袋中的误触等;
    \item 由点击精度引起的目标选择的误触,例如打字、手指移动光标等。
\end{itemize}

最后一种情况的优化涉及到针对应用的先验模型、用户心理模型等知识,而与手部行为无直接关系,故在此不做详细讨论。而针对前三种情况,我们希望在已知手部行为的图像信息的情况下,这三类误触问题均能得到较好的解决。

为了实现这三类误触问题的检测,从直接获取的手部行为信息的角度,我们需要检测以下三种信息:1、用户是否持握手机(是否正在使用),握姿如何;2、某次电容信号的产生是否由一个“指尖下落的过程”完成;3、对于一次有用户有意识的手部点击,哪些触点是用户“有意识点击”产生的,哪些是由误触产生的。

对于上述的三类信息,均可以通过\ref{cha:method}描述的手部全感知流程中直接或间接获取。

对于用户是否持握手机检测以及握姿检测,使用MobileNetV2为前馈特征提取网络的分类器进行分类。特别地,与\ref{cha:grip}中的握姿分类不同的是,目标类别中需加入一个无手部持握的图像类别(或纯环境类别),相应的需对该类别的数据进行采集。

对于指尖下落过程的检测,无论是对于RGB图像还是IR图像,首先通过\label{cha:fingertip}指尖检测算法检测出每一帧图像的所有指尖候选点,然后对于视频流所对应的指尖候选点集序列应用下降模式匹配\label{cha:motion}方法。若对于连续的k帧,最长下降序列长度为k',满足$k'>\lambda \cdot k$,其中$\lambda$为置信度阈值,则可判断此时有指尖下降行为发生。

对于触点检测,我们综合了握姿检测、指尖检测与原始图像经过MobileNetV2后输出的特征向量三种信息进行判断。我们将握姿解码成one-hot向量,将指尖标注的图像(仅有对应位置的标注,无原始图像)降采样后展平成一维向量,然后将他们与原图像的特征向量进行拼接,随后通过一个简单的全连接层线性分类器对其进行分类。在此,我们将触点检测任务分为对比触点检测与独立触点检测。对比触点检测需要在一对触点(其中一个为正确触点,一个为错误触点)中选择正确触点;独立触点检测需要判断给定的一个触点是否为正确触点。显然,后者的难度要大于前者:对于前者,我们希望模拟的是边缘误触的情况;对于后者,我们希望模拟由屏幕上的干扰信号引起误触的情况。

最终,我们完整的误触检测逻辑与流程如图\ref{fig:mistouch_flow}所示。

\begin{figure}[h]
  \centering
  \includegraphics[width=6in]{figures/method/mistouch_flow.pdf}
  \caption{误触检测流程图}
  \label{fig:mistouch_flow}
\end{figure}

\section{免验证检测}

智能手机认证与锁定作为一次自然的智能手机交互的起始步骤与结束步骤,其流程的合理性对安全性以及用户操作智能手机的效率与体验有着极其重要的影响。前人针对这个问题也进行了不同角度的研究。研究表明,智能手机用户平均花费约 2.9 \%的时间用于认证(最多的用户高达9\%),且用户认为在24.1\%的情况下安全性认证是不必要的\cite{185310}。基于上述事实,我们认为,更合理的智能手机认证与锁定流程能有效提升用户的使用效率与体验。现有智能手机的普遍锁定流程是:等待一段时间,若用户在该时间段内无与屏幕的交互,即电容屏感知不到触摸信号,则智能手机自动息屏并锁定。然而,这样的认证-锁定逻辑是不够智能的,因为用户在使用手机时,常常会有“非接触”且“不锁定”的需求。例如,用户需要以不确定的频率与手机进行频繁的交互,一个实际的场景是用户在与其他人通过社交软件进行多轮的聊天对话,此时,频繁的锁定-解锁过程是低效而繁琐的。我们针对上述情景,基于我们的手部全感知技术,提出了一种更高效的锁定-解锁流程:后台连续对当前手部状态进行检测,若为Hand-On或Hand-Around,即用户正在使用手机或在手机周围,则智能手机不主动锁定,当检测结果稳定为Hand-Off,即用户离开手机,则触发普通的解锁流程。我们它命名为免验证检测技术。免验证检测技术的目标是使智能手机能够更好地理解用户的非接触时行为,即长时间未触碰屏幕情况下的状态与意图,例如,用户是否在之后有交互需求,这有助于其进行更智能的反馈。

为了实现免验证检测技术,我们首先需要通过摄像头捕获的单目图像,区分以下三种模态:Hand-On,用户持握手机或正在操作手机,摄像头能捕获到近场的手部图像;Hand-Around,用户的手在手机周围,摄像头能捕捉到远场手部图像;Hand-Off,用户离开,摄像头捕捉不到手部图像。为此,我们需要对单目图像进行Hand-On、Hand-Around、Hand-Off三个模态的分类,这同样可以通过训练以MobileNetV2为前馈特征提取网络的分类器\ref{cha:grip}来完成。

在实现上述分类的基础上,我们引入“待锁定”状态作为智能手机“使用中”与“锁定”两个状态的中间状态,并为我们的免验证检测交互设计了解锁-锁定状态转移流程图(如图\ref{fig:hand_around})。

\begin{figure}[h]
  \centering
  \includegraphics[width=3in]{figures/method/HandAround.pdf}
  \caption{Hand-Around状态转移图}
  \label{fig:hand_around}
\end{figure}

进一步地,我们可以对我们的免验证技术进行更人性化的扩展,使其能够带来自然、高效、安全的交互体验。免验证技术本质上是通过区分Hand-On、Hand-Around、Hand-Off三种模态,感知用户手部的“非接触式”行为以及用户此时的状态,这一方面可用作安全性验证及相关决策,例如判断用户是否在周围、是否需要使用手机、手机是否有必要锁定等。另一方面,从用户的角度来看,他们对这项技术的期待,往往是在确保安全性的同时,获得交互与信息获取的效率与体验的优化,这启示我们从信息流的角度对免验证技术进行扩展,即在Hand-On、Hand-Around、Hand-Off三种模态下,在使用者不同的动作下,例如手在空中摇晃、手接近手机、手准备拿起手机等,手机能理解用户的交互需求,并对使用者有不同的信息反馈。例如当手机放在桌面,用户的手接近或准备拿起手机时,手机会自动亮起,在屏幕上显示出关键的信息,在用户拿起手机准备交互时,提前启动脸部识别完成验证——这极大地增加了用户日常交互流程的流畅性与效率。

\section{交互实现中的算法评测}

\subsection{下降检测}
对于误触识别中的下降检测,我们首先在\ref{dataset}中的IR数据集上运行指尖下降检测算法,我们从所有视频中选取三个最有代表性的视频:单手拇指操作、双手拇指操作、单手持握另一只手操作。这三个视频涵盖了用户日常操作手机的大部分握姿与手势。我们在每个视频中随机选取100-150个0.5s的视频区间,人工标注该区间下降检测结果与实际行为是否一致。

对于单手拇指操作、双手拇指操作、单手持握另一只手操作三种握姿对应的预测一致性的人工标注,正确数目、错误数目与正确率如表\ref{tbl:fingertip_movement}。

\begin{table}[htbp]
\centering
\caption{指尖下降检测准确率}
\label{tbl:fingertip_movement}
\begin{tabular}{p{130 pt}|p{60 pt}p{60 pt}p{60 pt}}
 \toprule
    & 正确帧数量 & 错误帧数量 & 正确率\\
 \midrule
    单手拇指操作 & 152 & 10 & 93.83\% \\
    双手拇指操作 & 92 & 7 & 92.93\% \\
    单手持握另一只手操作 & 93 & 13 & 87.74\% \\
\midrule        
    总计 & 337 & 30 & 91.83\% \\
\bottomrule
 \end{tabular}\\[2pt]
\end{table}

由以上结果可知,对于单手拇指与双手拇指操作的情况,下降检测的正确率分别可达93.83\%和92.93\%;单手持握另一只手操作的情况下,由于图像中手部行为与指尖行为更复杂,正确率稍有降低,但仍可以达到87.74\%。三种情况的总计正确率可达91.83\%,这说明我们的下降检测算法能十分有效地检出指尖下降行为。

\subsection{触点分类}

对于误触识别中的触点分类,我们分两类情况进行评测。第一类的评测数据集均采自真实误触场景,用于模拟手部边缘误触的情况:实验者依次点击屏幕上绿色的小球进行消除,在整个过程中,屏幕上所有触点将会被记录,我们将绿色的小球的坐标标记为非误触点,其他触点坐标标记为误触点,每组数据由一个正确的触点、一个同一时刻产生的误触点和该时刻的手部图像构成。第二类的评测数据集来自正确的触点与随机生成的误触点(与正确触点距离大于一定值),用于模拟屏幕上有其他干扰信号的情况,每组数据由一个触点(正确触点或错误触点)和该时刻的手部图像构成。评测任务是,在上述两种情况中,对于第一种,区分哪个触点是正确触点;对于第二种,区分改触点是否为正确触点。

实验结果显示,在真实落点的对比触点检测下,我们的方法正确率可以达到99.3\%;在模拟落点的独立触点检测下,我们的方法正确率可达90.8\%。这说明对于单手操作产生的边缘误触触点,我们的算法几乎可以完全消除这一类误触点;对于模拟干扰信号产生的误触点,我们的算法大概率(大于90\%)可将其消除。

\subsection{Hand-Around检测}

对于Hand-Around检测,我们在RGB数据集上进行Hand-On/Hand-Around/Hand-Off三分类与Hand-On/Hand-Off两分类的评测,分别测量每种情况的分类正确率。所有的模型与实验设定同握姿分类,前馈特征提取模型采用MobileNetV2,训练集/测试集按照1名实验者/1名实验者划分。

我们的方法在Hand-On/Hand-Around/Hand-Off 3分类任务上正确率可达94.36\%;在Hand-On/Hand-Off 2分类任务上正确率可达99.37\%。这说明我们的硬件与检测算法能够十分有效地实现用户手部Hand-On/Hnad-Around/Hand-Off三个模态的感知与分类。高准确率的Hand-Around分类信息可直接应用于交互中,增强用户的交互体验。

\section{用户实验}

在这一节,我们主要针对误触检测与免验证检测两个被动交互应用,从用户的角度出发,对问题发生频率、用户体验影响、用户效率影响、解决方案的价值这四个维度进行评价。我们希望通过用户实验来获取用户对于误触检测与Hand-Around检测的主观反馈。

我们通过问卷对20名实验参与者(12男8女,年龄20-30,拥有大于5年的电容屏智能手机使用经历)进行调查。

\subsection{常见误触场景的观点调查}

针对误触问题,我们列举了7种日常生活中使用手机时会出现误触的场景,它们覆盖了方法中描述的三类误触问题(如表\ref{MistouchTable})。对于每一种误触场景,我们请参与者就发生频率(参考:7分 每天>10次,6分 每天5-10次,5分 每天1-5次,4分 每周1-5次,3分 每月1-5次,2分 每年1-10次,1分 从来没发生过)、体验影响(参考:7分 影响很大,1分 几乎无影响)、效率影响(参考:7分 影响很大,1分 几乎无影响)进行7分Likert量评分,其中每种评分对应一个程度相关的陈述;紧接着,我们向他们描述针对该误触场景的解决方案所能达到的理想效果,请他们用7分Likert量评价该解决方案的价值(或者说对他们体验/效率的提升程度,或者他们是否想要这样的技术/体验)。

图\ref{fig:mistouch}展示了用户在七种误触场景下对发生频率、体验影响、效率影响、愿意使用四个维度进行的评分,其中1-7分别对应\ref{MistouchTable}中对应编号的场景。

\begin{table}[htbp]

\centering
\caption{常见的智能手机误触场景}
\label{MistouchTable}
\begin{tabular}{p{40 pt}p{280 pt}}
 \toprule
序号 & 场景 \\
 \midrule
1 & 单手使用手机时发生边缘误触 \\
2 & 单手使用手机时去“够”屏幕上够不着位置时发生的误触 \\
3 & 手机放在口袋中发生误触 \\
4 & 湿着手使用手机时操作不灵敏或误触 \\
5 & 下雨的时候在室外使用手机时操作不灵敏或误触 \\
6 & 手机息屏时握着手机不小心把屏幕点亮 \\
7 & 看视频时手不小心碰到手机弹出菜单 \\
\bottomrule
 \end{tabular}\\[2pt]
\end{table}

\begin{figure}[h]
  \centering
  \includegraphics[width=6in]{figures/experiment/mistouch.png}
  \caption{日常误触场景调查}
  \label{fig:mistouch}
\end{figure}

结果显示,对于我们提及的7种误触场景,发生频率的平均得分为3.86,最高得分为4.6,这说明这7种误触场景发生频率较频繁,平均每周内都会时有发生,其中“湿手操作手机不灵敏”与“单手‘够’屏幕够不着的位置”两种情况发生频率最高。除此之外,结果显示,7种误触场景在不同程度上对用户的体验与效率造成了一定影响,且用户们均希望(5分)这些误触问题能够得到解决。

\subsection{防误触系统体验实验}

我们邀请其中的5位参与者在实验设备上进行小球消除的任务:屏幕上的不同位置会依次出现绿色的小球,当参与者点中当前小球的位置时会出现下一个新的小球,在误触模式下,只有在无误触前提下点击才有效,若发生误触,则会出现一个弹框,参与者需要将其点击消除才能继续完成后续任务;在误触消除模式下,若有多个触点,则误触算法会自动抑制误触触点。参与者需要进行一次误触模式的任务与一次误触消除模式的任务,需要在规定的时间(2min)内完成尽可能多次的点击,他们的点击次数、误触次数以及两次任务的主观愉悦度(1-7分)将被记录。

对于小球消除任务,5名实验者在误触模式下的平均消除个数为152,平均误触次数为21,平均主观愉悦度为2.2;在误触消除模式下,平均消除个数为198,平均主观愉悦度为4.8。这说明对于单点点击任务,频繁的误触对操作效率与主观愉悦度均有较大的影响,这体现了我们的误触检测应用的价值。

\section{免验证检测的观点调查}

\begin{table}[htbp]
\centering
\caption{Hand-Around观点}
\label{HandAroundOp}
\begin{tabular}{p{40 pt}p{200 pt}}
 \toprule
序号 & 观点 \\
 \midrule
1 & 我愿意使用这种技术 \\
2 & 我担心这种技术的安全性 \\
3 & 我觉得这种技术能提高我的效率 \\
4 & 我觉得这种技术能带给我更好的体验 \\
5 & 这种技术在指纹解锁的场景下很有用 \\
6 & 这种技术在脸部解锁的场景下很有用 \\
7 & 这种技术在虹膜解锁的场景下很有用 \\
8 & 这种技术在密码解锁的场景下很有用 \\
\bottomrule
 \end{tabular}\\[2pt]
\end{table}

针对免验证检测应用,我们首先调查参与者的日常生活中的手机解锁频率、手机解锁方式、以及他们对当前手机认证方式效率的评价。然后我们向用户展示了8种关于解锁场景、解锁效率以及免验证检测应用的观点,请他们用7分Likert量表示对该观点的认同程度,观点如表\ref{HandAroundOp}。除此之外,我们还调研了用户对于Hand-Around感应(即手部靠近手机时手机显示一些信息)的交互想法的接受度,以及询问他们愿意以Hand-Around感应的形式获取哪些信息。

关于现有解锁方式效率,有50\%(10名)的实验者认为现有解锁方式能感觉延迟,但是仍能接受;有25\%(5名)的实验者认为现有解锁方式有明显的延迟,且这种延迟已经对操作体验造成了不同程度的影响。其中后者日常使用的解锁方式包括密码解锁与面部解锁两种,密码解锁的延迟来源于繁琐的操作过程,面部解锁的延迟来源于相对较高(与指纹相比)的失败率与失败代价(失败若干次需使用密码解锁)。我们还调查了实验者对于免验证检测应用的观点\ref{HandAroundOp}的认可程度。结果表明,用户们普遍愿意接受这样一项技术(4.85分),对于上述认为现有解锁方式有明显的延迟的实验者,他们的平均接受程度更是高达7分,但是一个普遍的担忧是这项技术的安全性问题(5.35分)。实验者认为,这项技术会在不同程度上提升自己的交互体验与交互效率,在密码解锁(4.65)与面部解锁(4.35)场景下体验与效率提升更明显,而在指纹解锁(3.5)上体验与效率提升相对有限。除此之外,实验者表示,对于免验证显示关键信息的功能,他们更希望从锁屏界面中获取天气、实时新闻、体育赛事等不涉及隐私的信息。