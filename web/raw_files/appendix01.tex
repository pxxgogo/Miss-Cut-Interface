\chapter{外文资料原文}
\label{cha:engorg}

\title{Air+Touch: 一种将触摸与空中手势交织的交互方式}

\textbf{摘要:}我们提出了Air+Touch,这是一种新的交互方式。它通过结合空中手势与触摸事件,提供一种统一的输入模式,其表现力大于每种单独的输入模态。我们展示了空中手势与触摸事件是如何高度互补的:触摸用于指定目标与切分空中手势,而空中手势则增加了触摸事件的表现力。例如,用户可以在空中绘制一个圆圈并点击以触发上下文菜单,在两次触摸之间进行一次“跳跃”以选择一个区域的文本,或拖动和在空中画“猪尾巴形”来将文本复制到剪贴板。通过观察性研究,我们根据空中手势是在触摸之前,之间还是之后发生,将Air+Touch 交互进行分类。为了说明我们的方法的潜力,我们构建了四个应用程序,展示了我们构建的七种Air+Touch交互方式。




\section{引言}

如今的移动设备依赖触摸作为主要输入模式。然而,指尖点击缺乏立即表达能力。为了支持更丰富的动作,触摸必须在时间(例如,长按)、空间(例如,绘制's'以使电话静音)或配置(双指点击是'alt click')的维度进行扩展。这些方法面临以下一个或多个问题:手势集的可扩展性,执行的耗时,“Midas”触摸以及小屏幕上的显著手指遮挡。因此,通过将触摸交互与新的输入维度相结合,扩展触摸交互是一个持续的挑战。最近,诸如三星S4等智能手机[22]等设备已经具备悬停感应功能。空中(或“自由空间”)手势是一个深入研究的领域(参见例如[7,17])。这些交互很有吸引力,因为它们可以利用比设备物理边界大许多倍的空间,从而实现更舒适且更有表现力的交互。然而,空中手势通常被视为单独的输入模态,而不是与现有的基于触摸的技术集成。此外,空中手势受到分割的挑战:很少有文献讨论过如何系统地区分有操作意图的手势与无意的手指运动。

\begin{figure}
\centering
\includegraphics[width=6in]{figures/reference/1.png}
\caption{我们提出了将触摸与空中手势交织,形成一种流畅、有表现力的交互方式}
\label{fig:ref_1}
\end{figure}

这篇论文中,我们重新考虑了触摸与空中手势,我们提出了一种综合上述两种输入模态的方法,使交互的语义丰富性与鲁棒性均高于单一模态。事实上,我们发现,空中手势和触摸操作是高度互补的:触摸用于指定目标与切分空中手势,而空中手势则增加了触摸事件的表现力。这样一种Air+Touch的模态刻画了一类允许用户在设备屏幕与其上方区域流畅操作的交互方式。

为了探索它的可能性,我们首先关注一种单一手指交互的情形——用户使用他的食指或拇指进行空中手势或触摸屏幕。通过观察,我们给Air+Touch交互设计了一个简单的分类:通过在触摸事件的之前、之间、之后插入空中手势来增强交互。反之,触摸事件可用来切分空中手势以及选定目标。对于空中手势,可以通过形状、速度、运动时间进行参数化。图\ref{fig:ref_1}提供了三个例子,从左到右分别是:(1)空中画圈后点击一个图标来处罚上下文菜单;(2)在两次触摸之间进行一次“跳跃”以选择一个区域内的文本;(3)点击后用手指在空中画圈可以持续放大地图。

\section{相关工作}

我们的工作将输入区域从触摸屏扩展到其上方的空间,这与触摸屏表面的侧面,后方和上方的交互研究相关。例如,SideSight [3]使用红外传感器跟踪移动设备侧面的手指移动。磁性传感器也被用于在Abracadabra [6]和MagiTact [14]中以实现类似的交互;Wigdor等人探索双面交互式桌面的设计[25];NanoTouch [2]和LucidTouch [24]证明了设备的背面可以用来扩展交互区域。

许多研究项目都集中在交互屏幕上方的空间,例如Hilliges等人的“空中交互”[7],Marquardt等人的“连续交互空间”[17]和Banerjee等人的研究项目[1]。在移动设备领域,HoverFlow使用红外传感器[1],Niikura等人使用高帧率摄像头[18]跟踪移动设备上方的手/手势。Marquardt等人提出将数字表面及其上方的空间结合,使得触摸、手势和物体均可触发交互[17]。然而,还没有工作讨论过用于分割空中手势的机制(例如:检测非交互意图的手指运动)。进一步的,(这些工作)交互空间和触摸手势通常共存,而不是像我们提出的那样相互结合。

自然地,研究人员下一步要研究的是设备周围的环绕空间中的手势交互。 Kratz等人指出与虚拟轨迹球相比,通过手势进行设备的环绕操作(移动设备上方,旁边和后方)在操纵3D对象时会产生更好的性能[15]。琼斯等人发现设备周围的自由空间交互可以像触摸一样好[13]。这项工作还定义了设备周围的“舒适区”,这对不同方向的传感器的应用具有重要意义。三星在Galaxy S4上发布了几款基本的空中手势[22]:一只手在锁定屏幕上移动显示时间和通知,并在屏幕上方向左或向右滑动导航相册。 

Air+Touch还基于以前的工作,这些工作综合了多种输入技术,实现新型交互的原型。 Pen + Touch [11]综合了笔和触摸输入形成一种新的输入模态,例如使用触摸来保持照片,使用笔拖动并创建副本。 Motion + Touch [10] 将触摸与运动感知结合,以产生触摸增强的动作手势和运动增强的触摸。Pen + Motion [9]将笔输入与笔的运动相结合,实现新的输入模态。我们的工作以几种新的方式综合了触摸和空中手势。首先,我们提供一种输入结构,使用触摸来分割空中手势,并使用空中手势来增强触摸。其次,空中手势和触摸相互交错,产生输入序列,给交互带来更多信息。

\section{交互设计的观察}

为了指导和指导我们对Air+Touch交互的初步探索,我们进行了一项研究,以便在用户参与交互式任务时观察移动屏幕上方的手指行为。我们招募了12名参与者(5名女性,年龄24-36岁)。一名参与者是左撇子,一名是双手,所有人都是普通的智能手机用户。

我们要求每个参与者在智能手机上执行一组常见任务(例如,撰写文本消息,在地图上导航)。我们对会话进行了录像,并寻找手指离开屏幕后的运动模式。由此,我们提炼出一组可以转化为手势输入的特征,同时避免与自然手指运动的碰撞(即减少混淆)。接下来,我们将讨论这些功能如何有助于空中手势的设计,以及如何将触摸用作自然分隔符来分割这些动作。 

\subsubsection{Air:屏幕上方手指的运动性质}

实验的参与者展示了一系列空中、屏幕上方的手指行为。这包括触摸之间的悬空、当屏幕需要被读出时收回手指至边框、不确定做什么时扭动手指(例如在找某个按钮)。在讨论屏幕内容时,人们还会用他们的手指指向内容或在内容处挥动;或者在他们说话时做出手势,就像日常对话中那样。特别地,我们关注手指运动行为的三个主要类别:1)路径 - 手指运动的轨迹,2)位置 - 手指在屏幕上方的特定位置,以及3)重复 - 用户重复某些手指运动。

这些观察结果以两种方式启发了Air + Touch设计。首先,这阐明了用户可以舒适地完成的屏幕上方的手指动作,我们将其作为我们的空中动作词汇集合的一部分。其次,它允许我们制作一个区别于自然手势的手势词汇表。以下是一些启示我们Air + Touch手势设计的手势示例:

\textit{椭圆路径}:我们观察到很少有手指运动遵循平滑的椭圆路径,这说明椭圆路径的环绕手势是可分辨的。

\textit{矩形路径}:与Grossman等人在[5]中的发现相似,很少参与者的手指轨迹中会出现直角。这表明带有棱角的路径,例如“L型”路径,可以被鲁棒地识别。


\textit{高度变化}:大多数用户的手指移动发生在靠近屏幕的位置。这表明具有明显高度变化的空中姿势与普通的交互手势是可分辨的。

\textit{使用框架手势}:Whack Gestures [12]证明,当作为框架特征时,简单的手势(例如,敲击)可以产生表达性输入。虽然底层的手势具有高错误率,但是匹配的框架手势可以显着降低误报的可能性。在我们的研究中,我们观察到用户很少两次触摸同一位置,除非滚动,并且在这种情况下,很少在中间时间执行任何待识别的手势。这表明可以在框架触摸之间执行空中手势。另一种可能性是在空中手势中包括框架,例如在空中画圈作为触发手指盘旋识别的信号(类似于[12]中使用连续性“击打”)。


\subsection{Touch: 限定空中手势}

即使使用精心设计的空中手势集合,自由空间手势的不确定性也需要显式的交互起始与终止信号。观察表明触摸事件可以作为一个强大而直观的分隔符。在典型的交互任务中,触摸与空中手势通过触摸引出空中手势与触摸终止空中手势两种方式结合。因此,触摸自然地将空中手势分成三类:触摸之前,之间或之后。这允许空中手势识别引擎仅在空中手指移动窗口时间内搜索 (而不必不断监测)。在本文的其余部分,这些时间类别作为Air + Touch交互示例的分类标准。


\subsection{Air+Touch: 手势词汇集合}

我们的观察也帮助我们创建了空中手势的初始词汇集合,如上节所述,可以通过三种方式触摸事件来界定。为了进一步探索其交互空间,我们观察了现有的应用程序,并考虑是否可以采用任何Air + Touch手势来增强当前的交互。这有助于我们提出四个应用程序,涵盖一组七个Air + Touch手势(图\ref{fig:ref_2},红色),这些手势是整个设计空间的代表(但不包括在内)。

\begin{figure}
\centering
\includegraphics[width=5in]{figures/reference/2.png}
\caption{一个原型验证的Air+Touch手势设计空间,我们实现了其中的7项技术 (红色阴影)}
\label{fig:ref_2}
\end{figure}

\begin{itemize}
    \item 角手势:手指在空气中绘制90度角(在垂直于屏幕的平面上)。
    \item 圆手势:手指在空中绘制平滑的圆形路径。
    \item 猪尾手势:手指沿着空中轨迹绘制一个小环。
    \item “之型”手势:手指在空气中(在与屏幕平行的平面上)产生剧烈的“转弯”,例如画出“L”或“Z”;
    \item 尖刺手势:手指在其移动期间到达空中特殊位置,例如,到达高于通常悬停范围的位置,或屏幕边界外的位置。
\end{itemize}

\section{AIR+TOUCH 原型}

越来越多的设备具有能够在空中跟踪手指的电容式触摸屏(例如,悬浮检测)。在2014年国际消费电子展上,Synaptics展示了一款原型触摸板,能够在距离最远4厘米处追踪手指[23]。所有迹象表明,这项技术将继续改进并变得更加普遍。不幸的是,当今消费类设备的感应范围有限。例如,三星Galaxy S4的跟踪范围约为1.5厘米。

因此,为了探索可能实现的各种Air+Touch交互,我们需要构建我们自己的原型。虽然体积庞大,但我们的原型仅用作探索和调查。我们还使用该平台构建了七个Air+Touch交互演示应用(图\ref{fig:ref_2}和图\ref{fig:ref_6}-12),这些演示涵盖了Air+Touch的设计空间并展示了我们方法的可行性。

\subsection{硬件设计}

我们的原型手指跟踪系统包括商用智能手机和倾斜安装在普通机箱上的PMD Camboard Nano [19]深度摄像头(图\ref{fig:ref_3})。 Camboard Nano具有90º×68º的视野,感应160×120px的深度和5至50厘米的红外图像,最高可达90 fps。手指跟踪在外部PC上执行,手指位置通过无线网络发送到移动客户端。这种设置使我们能够快速制作创意原型,而无需在智能手机中安装任何定制硬件。

\begin{figure}
\centering
\includegraphics[width=6in]{figures/reference/3.png}
\caption{我们的原型硬件设计}
\label{fig:ref_3}
\end{figure}

\subsection{手指追踪}

我们的手指跟踪软件使用C++编写,并使用OpenCV库。由于手机的几何结构已知,我们可以通过简单的基于体积的背景减法(图\ref{fig:ref_4}b)。我们还消除了手机屏幕红外反射引起的噪音(图\ref{fig:ref_4}c)。使用此图像,我们识别场景中的最大斑点并进行轮廓分析。我们假设指尖是blob质心最远的轮廓点(图\ref{fig:ref_4}d)。为了防止误报,我们只查看沿着手指主要方向的轮廓。

\begin{figure}
\centering
\includegraphics[width=6in]{figures/reference/4.png}
\caption{我们的手指识别流程:a)深度图;b)背景移除;c)轮廓检测;d)指尖检测}
\label{fig:ref_4}
\end{figure}

在手指指向深度相机的情况下,指尖不会沿着轮廓,而是位于手指边界内。我们使用相机的红外图像检测到这种情况;由于皮肤的高红外反射率(以及我们的深度相机使用的红外发射器),指尖将显示为明亮的粗糙高斯光斑。在这种情况下,我们使用最亮点作为指尖位置。

这个过程产生一个相机空间,指尖X/Y/Z位置,表示Air+Touch手势期间的兴趣点。然后,我们将此原始3D坐标转换为X/Y屏幕坐标(以像素为单位),以及Z值(垂直于屏幕的距离)。该变换矩阵使用手机屏幕上的三个已知点计算,在一次校准过程中在3D相机空间中选择(图\ref{fig:ref_4},空心点)。最后,使用指数加权移动平均对指尖位置进行平滑。

\subsection{空中手势分类}

我们的系统以每秒20帧的速度记录3D手指位置,并保持大约一秒的位置历史记录。当发生触碰事件时,我们在缓冲手指位置的X和Y坐标(投影到屏幕空间上)上运行 \$ 1手势识别算法[26]。

如果找到具有足够大小的良好形状匹配,则触发相应的交互事件。对于触摸后的空中手势,我们在触摸事件后大约一秒钟后在缓冲器上运行识别算法。在触摸时,我们还会检查过去一秒内是否有触摸时间,如果是,则将其解释为在两个触摸事件之间执行的空中手势。

为了支持利用Z距离(而不是形状)的空中手势,我们使用位于屏幕上方4cm处的虚拟平面作为阈值,类似于3D交叉手势。每次越过该平面时,都会记录时间戳。如果触摸事件发生在500ms内,则会触发交互事件。

\section{AIR+TOUCH交互技术实例}

基于我们观察研究的结果,我们开发了一组Air+Touch示例交互(图\ref{fig:ref_5})。为了提供这些交互技术的使用场景,我们创建了四个应用程序:照片查看器,绘图应用程序,文档阅读器和地图。另请参阅我们的视频图。

\begin{figure}
\centering
\includegraphics[width=6in]{figures/reference/5.png}
\caption{Air+Touch交互分类}
\label{fig:ref_5}
\end{figure}

\subsection{Air Before Touch}

与鼠标不同,触摸(通常)只有一个“按钮”。这导致了额外的模态机制的需求,例如触摸并保持以调用例如上下文菜单。工具栏也是一个常用场景,但却占用了宝贵的屏幕空间。为了缓解这个问题,Air+Touch允许用户在触摸屏幕之前或途中执行空中手势,作为参数化触摸事件的方式。我们为此技术提供了两个示例交互。

\subsubsection{盘旋触碰}

在我们的照片查看器应用程序(图\ref{fig:ref_6})中,用户可以通过在点击所需图像之前执行空中盘旋运动(图\ref{fig:ref_6}a-c)来触发图像的上下文菜单(图\ref{fig:ref_6}d-e)。空中手势指定命令(在这种情况下,触发上下文菜单),而触摸指定感兴趣的项目(例如,照片)。这两个动作组合成一个流畅的手指动作:盘旋触碰。

\begin{figure}
\centering
\includegraphics[width=6in]{figures/reference/6.png}
\caption{在我们的图片浏览器应用中,空中环绕并触摸触发上下文菜单}
\label{fig:ref_6}
\end{figure}

\subsubsection{用于模式切换的“高跳触碰”}

当只有拇指可用于交互时,单手地图导航在移动手持设备上是困难的。我们的地图应用程序演示了Air + Touch如何让用户在点击和缩放模式之间切换,只需在点击之前将拇指“抬高”(图\ref{fig:ref_7})即可。然后,该人可以在屏幕上滚动以平移地图(图\ref{fig:ref_7}ab),或者放大/缩小地图,就像使用虚拟滑块(图\ref{fig:ref_7}cd)。

\begin{figure}
\centering
\includegraphics[width=6in]{figures/reference/7.png}
\caption{在地图应用中,通过指尖的“高抬”实现模式的切换}
\label{fig:ref_7}
\end{figure}

\subsection{Air Between Touch}

在连续触摸事件之间执行空中手势提供了参数化两点或甚至多点动作的可能。

\subsubsection{在触摸之间通过手指“高跳”来选择文本}

由于没有直接的方法来消除触摸界面中滚动和选择之间的歧义,所以诸如复制和粘贴之类的常规操作是不实用的。 Air + Touch可以通过一个两步的方案简化这一过程(图\ref{fig:ref_8})。用户可以通过以下步骤选择文本区域:1)轻敲所需选定区域的开头,2)将手指抬高,然后3)触摸所选区域的末端。按顺序,这三个步骤可以在单个手指运动中执行。如果需要,进一步的接触可以提供细粒度的调整(d)。这创建了一个手势快捷方式,将文本区域的规范和选择它的意图分成[4]单手指'high-jump'。

\begin{figure}
\centering
\includegraphics[width=6in]{figures/reference/8.png}
\caption{在我们的阅读器应用中,触摸之间通过手指“高跳”来选择文本区域}
\label{fig:ref_8}
\end{figure}

\subsubsection{在触摸之间画“L”形进行滚动文字的选择}

类似地,裁剪或选择图像的子区域通常需要首先中断当前交互,然后指定特殊应用模式(例如,通过工具栏按钮)。然而,使用Air + Touch,这可以通过在两次触摸之间执行“L”手势以更流畅的方式实现。第一和第二触摸指定矩形选框的相对角。在驾驶中,我们发现绘制“L”是表达选择矩形区域意图的简洁而自然的方式。

\subsection{Air After Touch}

在该类别中,当手指离开表面时,人进行空中手势。空气通过将触摸映射到特定功能(类似于触摸前的空气)或通过允许触摸继续不受屏幕尺寸限制的交互(例如,无离合器滚动和缩放)来增强触摸。

\subsubsection{触摸后画“尾纤”触发自由选择}

在我们的绘图应用程序中,在屏幕上拖动手指用于绘制。但是,可以使用触摸后空中手势来参数化此路径。例如,通过抬起手指并在空中进行尾纤运动(图\ref{fig:ref_10}),最后绘制的路径被转换成可以例如移动,缩放或复制到剪贴板的裁剪区域。

\begin{figure}
\centering
\includegraphics[width=6in]{figures/reference/10.png}
\caption{在我们的绘图应用中,用户通过“尾纤”剪切区域}
\label{fig:ref_10}
\end{figure}

\subsubsection{触摸后空中环绕触发地图缩放}

我们之前描述过触摸前手势技术,可以在平移和缩放之间快速切换模式。另一种解决方案是“分工” - 触摸可用于平移,而空中环绕用于缩放。更具体地,一个人通过点击例如地图来开始以指定缩放中心(a)。当她从屏幕上松开手指时,可以通过高空中画圈触发“缩放模式”(图\ref{fig:ref_11}b)。一旦处于“缩放模式”,手指在空中连续旋转用于放大或缩小(取决于旋转方向方向)(图\ref{fig:ref_11}cde)。点击屏幕或短时间的非环形动作会退出缩放模式。这种技术利用了重复手势的概念;即使手指在触摸后意外地在空中画圆,也会在最坏的情况下打开变焦模式但不会导致任何实际的缩放。

\begin{figure}
\centering
\includegraphics[width=6in]{figures/reference/11.png}
\caption{在我们的地图应用中,用户通过高空中画圈触发“缩放模式”}
\label{fig:ref_11}
\end{figure}

\subsubsection{触摸后悬空控制滚动速度}

在触摸屏上,由于触摸受到屏幕物理表面的限制,因此不可避免地会发生“clutching”。例如,滚动长页面需要重复手指摆动[20,21]。我们的阅读器应用程序可以精确控制长列表的页面滚动。当用户触发时
通过摆动控制(a)惯性滚动,他可以使用手指的悬停高度来控制滚动速度 - 更高的手指位置映射到更快的滚动(图12b-d)。这类似于Zliding和Zoofing技术[20,21],但使用Z距离代替压力。触摸屏幕停止滚动。两个高度阈值用于区分这种“悬停滚动”与正常滚动,这不受影响。

\begin{figure}
\centering
\includegraphics[width=6in]{figures/reference12.png}
\caption{在阅读器应用中,通过悬空高度控制滚动速度}
\label{fig:ref_12}
\end{figure}

\section{讨论}

我们展示的交互实例只是所有可能的交互应用中的一个很小的子集,我们相信它们展示了Air+Touch的表现力与潜力。更重要的是,Air+Touch动作可以与传统的触摸手势(例如平移和点击、缩放、组合滑动等)协同工作。正如我们的观察研究所强调的,并在我们的示例应用程序中实现的那样,Air+Touch技术可以在触摸事件之前、之间和之后交织空中手势。通过广泛的使用和试验,很明显这些分类具有不同的优势,可以支持各种交互式任务:

\begin{itemize}
    \item 触摸前和触摸后的空中手势都能使快速模式切换与手指的下降、上升动作相映射(如图\ref{fig:ref_6})。它们还可以将一个动作指派给特定于一组接触点的操作(例如,图\ref{fig:ref_10});
    
    \item 触摸后的空中手势进一步地允许用户连续地进行下一个以触摸操作开始的操作;
    
    \item 触摸之间的空中手势在需要跨屏幕上多位置的操作的任务上具有天然的优势。空中手势可以被嵌入到两次触摸事件之间,这节省了工具切换或模式切换的时间开销(例如,图\ref{fig:ref_9})。
\end{itemize}

\begin{figure}
\centering
\includegraphics[width=6in]{figures/reference/9.png}
\caption{通过"L"型空中手势进行区域选择}
\label{fig:ref_9}
\end{figure}

\subsection{将空中手势与触摸划分为流畅的交互序列}

Table 1提供了一个Air+Touch技术在六个交互任务中与传统触摸交互方式的比较。虽然这些任务的基本元素相同(例如,文本选择由指定选择模式和指定选择区间组成),传统触摸交互方式按照离散化的步骤将他们完成,而Air+Touch将这些元素截断并组织成流畅的交互。对于熟练的用户,他们可以很好地将Air+Touch交互技术与他们的日常交互相结合,构成一个流畅动作;而对于传统触摸交互方式它仍然是离散的动作序列。

\begin{figure}
\centering
\includegraphics[width=6in]{figures/reference/13.png}
\label{fig:ref_13}
\end{figure}

\subsection{根据伴随的触摸进行合理的空中手势选择}
我们最初触发上下文的概念是通过绘制一个“尾纤”然后点击相应的目标。然而,我们发现,这非常难实现,因为当手指绘制“尾纤”后,它会偏离原来的目标,这需要用户在空中手势的最后重新定位。对比来看,一个“完整的圆”的空中手势就相对容易,因为手指可以自然地环绕一圈,然后回到它的最初位置,在该位置用户可以很自然地下降点击目标。相反地,当设计一个触摸后的空中动作,我们发现“尾纤”是很容易实现的,因为它没有结束位置的限制。这表明空中手势的选择需要考虑与之前、之后触摸动作衔接的自然性。


\subsection{在触摸前与触摸后分割空中手势}

对于触摸前或触摸后的空中手势,触摸仅仅分割了空中手势的开始或结束部分,这留给开发者更多的可能性来决定何时开始/结束处理手指剩余的运动。这可转化成实现层面上的问题,即缓存区域需要设置多大以保存指尖的历史3D位置信息。在原型设计中,我们可视化了手指的轨迹并把它投影到屏幕上。我们选择了一个合适的缓冲区大小,使得既不会捕获不完整手势,也不会捕获过多的帧数。另一个可行的方法是分析分析不同大小的缓存区,并选择手势识别置信度最高的缓存区大小。

\section{总结}

悬空交互技术在2014年CES的普及以及在旗舰设备(如即将发布的Galaxy S5)中的持续应用表明,空中手势技术将持续走向成熟,并可能在触摸设备中发挥越来越大的作用。如今,只有很少的“空中”手势得到了支持,并从根本上从触摸互动中分离出来。我们的工作有助于通过协同交织的触摸和空中手势,组合为更强大的交互模式。一方面,空气手势增强触摸,增加表现力;另一方面,触摸分割空气手势,以解决分割模糊性问题。有了良好的设计,这些动作可以融合成单一的、流畅的动作,带来了每种独立形式都无法达到的表现力。尽管如此,仍有许多未来的工作需要考虑,包括扩大手势词汇集合,捕捉3-自由度的位置,与3-自由度的手指旋转,以及多指交互问题。

\section{书面翻译对应的外文材料原文索引}
\begin{translationbib}
\item Chen X A, Schwarz J, Harrison C, et al.  Air+touch: Interweaving touch and in-air gestures UIST ’14: Proceedings of the 27th Annual ACM Symposium on UserInterface Software and Technology. New York, NY, USA: ACM, 2014: 519–525.http://doi.acm.org/10.1145/2642918.2647392
\end{translationbib}
