\chapter{引言}
\label{cha:intro}

\section{研究背景}
随着信息技术的不断发展,如今,智能手机已经成为人们生活中必不可少的移动电子设备。统计\footnote{Statista. 2018. Daily time spent on mobile by Millennial internet users worldwide from 2012 to 2017 (in minutes). Retrieved January 7, 2018, from https://www.statista.com/statistics/283138/millennials-daily-mobile-usage/ }表明,青年人群(1980年-2000年之间出生)每日平均花费在移动互联网应用上的时间在2012-2017年间从107分钟提升到了223分钟,这足以说明智能手机在人们日常生活中的重要作用。

同时,智能手机也是日常生活中人们最频繁的交互设备之一,对交互效率、性能与交互体验也有相对较高的要求。从第一代按键式手机的出现开始至今,智能手机的输入方式及用户交互方式也在经历着不断的演变与更迭。传感媒介从物理按键到电阻触摸屏到电容触摸屏,交互功能逐渐趋于复杂化与智能化。发展至今,电容屏幕的触摸信号成为当今智能手机输入的主要模态。然而,单一的电容信号输入存在许多问题与交互的不便。例如:

\begin{itemize}
    \item \textbf{误触问题:} 误触是人们使用智能手机时频繁出现的问题,受限与屏幕大小、屏幕传感原理等因素,误触在人们使用智能手机时频繁发生,极大地影响了智能手机的使用效率与使用体验。在电容屏幕的传感模态下,手机直接接收的输入信号为二维电容信号序列,因此无法有效理解当前信号是否由有输入意图的用户行为产生,造成误触。
    \item \textbf{操作的自然性与舒适性问题:}用户的操作舒适性会受到设备屏幕大小、用户界面设计等因素影响。用户在不同姿态下手部舒适区域根据屏幕大小发生变化\cite{Le:2018:FRC:3173574.3173605},因此,针对手部舒适区域进行界面布局与交互方式的优化能有效地提升操作体验。相反,不合理的界面布局或交互设计会造成用户持握姿态的频繁变换\cite{Eardley:2017:UGS:3025453.3025835} ,造成体验上的不佳。
    \item \textbf{交互效率问题:}单一的触摸信号能传达的信息十分有限,限制了用户的交互效率。若要表达更丰富的信息,需要在时间、空间维度对触摸信号进行组合与扩充,但这些扩充仍然受限于延迟、手势集的可扩展性等问题\cite{Chen:2014:AIT:2642918.2647392}。
\end{itemize}

针对上述问题,现有的许多工作尝试探索如何克服上述的局限性,并尝试探索超越传统触摸交互的新型交互方式。例如:通过区分手指与握姿增强交互\cite{Masson:2017:WIF:3126594.3126619}\cite{Lim:2016:WAR:2957265.2961857},根据握姿自适应的布局调整\cite{Lim:2016:WAR:2957265.2961857}\cite{Cheng:2013:IAS:2470654.2481424},将屏幕的上方\cite{Chen:2014:AIT:2642918.2647392}\cite{Hasan:2016:TST:2983310.2985755}\cite{Hinckley:2016:PSM:2858036.2858095}、侧面\cite{Chang2006Recognition}\cite{Cheng:2013:IAS:2470654.2481424}\cite{Cheng:2013:IGA:2468356.2479514}、背面\cite{Corsten:2017:BUB:3025453.3025565}\cite{Wong:2016:BBO:2999508.2999522}的空间扩展为交互空间等。

然而,上述绝大多数的交互设计均没有很好地考虑到在实际应用场景中引入额外的传感器带来信息增益与额外的硬件成本以及体积、便携性、实用性等负面影响之间的权衡。因此,这些设计容易走入两个极端:
\begin{itemize}
\item \textbf{完全不引入额外的感测设备与传感信息:}这一部分工作希望从现有传感设备所捕获的原始数据中进一步地挖掘出关键信息(例如通过屏幕电容信号的形状检测误触、电容信号与IMU(运动传感器)数据结合实现交互增强)。其局限性在于,虽然其充分地挖掘与利用了已有的传感信息,但是客观上其获取信息是不完整的,有些模态的信息是不能通过内置传感器直接或间接获取的(例如图像、屏外手部动作),因此这部分信息无法被用来支持交互。

\item \textbf{无限制地引入传感设备:}这一部分工作从交互需求触发,通过额外的外置传感器,例如外置深度摄像头、环绕式灵敏电容传感器、声学传感器等,来捕获交互所需要的传感数据。其局限性在于,引入额外的感测系统或额外的传感器会带来额外的硬件成本以及体积、便携性、实用性等负面影响,因此多数增加外置传感器的实验系统仅能用于验证交互设计。

\end{itemize}

HandSee\cite{Yu:2019:HEF:3290605.3300935}则在上述两种极端之间找到了一个很好的平衡:它通过在手机前置摄像头上方放置一面三棱镜,原像与镜面反射像构成有一定视差的双目视觉系统,使其能够捕捉用户操作手机时的手部深度图像,并进一步地基于此完成一系列应用。HandSee利用手机内置硬件,通过简单的改造,大幅增强手机对用户交互行为的感知能力,为实现更智能、用户体验更好交互提供了信息感知的支持。然而其局限性有三点:一是其运算流程在服务器端使用GPU完成,流程设计未充分考虑计算量与计算效率问题;二是HandSee的深度图重构部分的条件是及其严苛的,它对三棱镜的位置的精密性、光照条件(光照充足、无太阳直射、无炫光)、反射面条件等均有较高的要求;三是其深度图重构部分噪音较大、指尖识别鲁棒性仍有欠缺。

因此,本工作的动机是对于全手交互\cite{Yu:2019:HEF:3290605.3300935}这样一种全新的交互理念,我们希望提出一个更高效、适用性更好的全手感知的流程。

\section{工作概述}

在本研究工作中,我们借鉴了HandSee\cite{Yu:2019:HEF:3290605.3300935}中借助三棱镜的轻量化硬件设计的想法,重新进行了硬件设计。新的硬件可模拟RGB摄像头与IR摄像头(红外摄像头)两种工作模式。

算法上,我们在该硬件方案的基础上提出了一套高效的在RGB与IR两种模态下的基于单目图像的智能手机手部感知的算法流水线。首先,我们使用MobileNetV2模型对视频流中的视频帧进行握姿分类,实时地检测用户当前的握姿。在RGB单目图像下,我们使用MobileNetV2-SSD模型进行指尖区域的检测;在IR单目图像下,我们提出了一种基于握姿与边缘曲率统计特征的轮廓分类与指尖识别方法。除此之外,我们还提出了一种有效的运动模式匹配与检测算法,能有效检测指尖某种模态的运动过程(如下落、上升)。

最后,基于该算法流水线,我们设计了两种被动交互的应用:误触检测与免解锁检测。我们创新性地通过握姿分类、下落过程检测、触点分类三个步骤有效地实现了不同场景下的误触检测;通过算法流程中的姿态分类实现Hand-On,Hand-Around,Hand-Off三种手部状态的分类。我们还通过定量实验分析与定性用户实验对两个应用的准确率、用户接受程度进行了评价。


\section{主要贡献}

本研究的主要贡献是提出了一套高效的在RGB与IR两种模态下的基于单目图像的智能手机手部感知的算法,并基于该算法提出了误触检测与Hand-Around检测这两个创新性的交互设计。
具体的贡献与创新点可概括为以下三点:
\begin{itemize}
    \item 提出了基于单目图像的智能手机手部感知的算法,算法流程计算效率高,能够支持移动端实时离线计算,且宽容性及场景适用性更好。
    \item 进一步探索了智能手机手部全感知技术在单目RGB与IR摄像头的硬件条件下的可能性,在RGB、IR环境下分别提出了更鲁棒、更高效的指尖识别解决方案,并基于指尖识别结果提出了一种手指运动模式检测的方法。
    \item 提出了误触检测与免验证检测两个被动交互的设计,并通过定量实验分析与定性用户实验证明这两种被动交互设计的有效性与优越性。
\end{itemize}

\section{论文结构}
本论文全文共分为6章,在本小节中,我们将对全文的整体结构进行简要的介绍。第一章中,我们先介绍了整个项目的研究背景以及现有工作的局限性,然后概括了本篇论文的研究工作的整体流程与主要贡献。第二章中,我们对智能手机手部交互增强、图像分类、物体检测、基于视觉的手部识别等相关领域的文献进行了调研与总结,并对不同工作的优缺点进行了分析与讨论。第三章中,我们介绍了我们的基于屏上摄像头的手型识别与交互技术的硬件设计方案。第四章中,我们详细地说明我们提出的RGB/IR双模态下基于单目图像的手部感知算法中每一步骤的原理,并对算法每一部分的效率与准确率进行评测。第五章中,我们介绍了基于单目手部全感知算法流程的两个被动交互设计,误触检测与免验证检测的设计背景与实现细节,并通过定量实验与用户实验对两个交互应用进行评测。第六章中,我们总结了全文,并对未来的工作进行展望。
